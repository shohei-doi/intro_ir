% Options for packages loaded elsewhere
\PassOptionsToPackage{unicode}{hyperref}
\PassOptionsToPackage{hyphens}{url}
\PassOptionsToPackage{dvipsnames,svgnames,x11names}{xcolor}
%
\documentclass[
  xelatex,
  ja=standard]{bxjsarticle}

\usepackage{amsmath,amssymb}
\usepackage{iftex}
\ifPDFTeX
  \usepackage[T1]{fontenc}
  \usepackage[utf8]{inputenc}
  \usepackage{textcomp} % provide euro and other symbols
\else % if luatex or xetex
  \usepackage{unicode-math}
  \defaultfontfeatures{Scale=MatchLowercase}
  \defaultfontfeatures[\rmfamily]{Ligatures=TeX,Scale=1}
\fi
\usepackage{lmodern}
\ifPDFTeX\else  
    % xetex/luatex font selection
  \setmainfont[BoldFont=Noto Sans CJK JP]{Noto Serif CJK JP}
\fi
% Use upquote if available, for straight quotes in verbatim environments
\IfFileExists{upquote.sty}{\usepackage{upquote}}{}
\IfFileExists{microtype.sty}{% use microtype if available
  \usepackage[]{microtype}
  \UseMicrotypeSet[protrusion]{basicmath} % disable protrusion for tt fonts
}{}
\makeatletter
\@ifundefined{KOMAClassName}{% if non-KOMA class
  \IfFileExists{parskip.sty}{%
    \usepackage{parskip}
  }{% else
    \setlength{\parindent}{0pt}
    \setlength{\parskip}{6pt plus 2pt minus 1pt}}
}{% if KOMA class
  \KOMAoptions{parskip=half}}
\makeatother
\usepackage{xcolor}
\setlength{\emergencystretch}{3em} % prevent overfull lines
\setcounter{secnumdepth}{5}
% Make \paragraph and \subparagraph free-standing
\ifx\paragraph\undefined\else
  \let\oldparagraph\paragraph
  \renewcommand{\paragraph}[1]{\oldparagraph{#1}\mbox{}}
\fi
\ifx\subparagraph\undefined\else
  \let\oldsubparagraph\subparagraph
  \renewcommand{\subparagraph}[1]{\oldsubparagraph{#1}\mbox{}}
\fi


\providecommand{\tightlist}{%
  \setlength{\itemsep}{0pt}\setlength{\parskip}{0pt}}\usepackage{longtable,booktabs,array}
\usepackage{calc} % for calculating minipage widths
% Correct order of tables after \paragraph or \subparagraph
\usepackage{etoolbox}
\makeatletter
\patchcmd\longtable{\par}{\if@noskipsec\mbox{}\fi\par}{}{}
\makeatother
% Allow footnotes in longtable head/foot
\IfFileExists{footnotehyper.sty}{\usepackage{footnotehyper}}{\usepackage{footnote}}
\makesavenoteenv{longtable}
\usepackage{graphicx}
\makeatletter
\def\maxwidth{\ifdim\Gin@nat@width>\linewidth\linewidth\else\Gin@nat@width\fi}
\def\maxheight{\ifdim\Gin@nat@height>\textheight\textheight\else\Gin@nat@height\fi}
\makeatother
% Scale images if necessary, so that they will not overflow the page
% margins by default, and it is still possible to overwrite the defaults
% using explicit options in \includegraphics[width, height, ...]{}
\setkeys{Gin}{width=\maxwidth,height=\maxheight,keepaspectratio}
% Set default figure placement to htbp
\makeatletter
\def\fps@figure{htbp}
\makeatother

\renewcommand{\thefootnote}{\arabic{footnote}}
\makeatletter
\makeatother
\makeatletter
\makeatother
\makeatletter
\@ifpackageloaded{caption}{}{\usepackage{caption}}
\AtBeginDocument{%
\ifdefined\contentsname
  \renewcommand*\contentsname{目次}
\else
  \newcommand\contentsname{目次}
\fi
\ifdefined\listfigurename
  \renewcommand*\listfigurename{図一覧}
\else
  \newcommand\listfigurename{図一覧}
\fi
\ifdefined\listtablename
  \renewcommand*\listtablename{表一覧}
\else
  \newcommand\listtablename{表一覧}
\fi
\ifdefined\figurename
  \renewcommand*\figurename{図}
\else
  \newcommand\figurename{図}
\fi
\ifdefined\tablename
  \renewcommand*\tablename{表}
\else
  \newcommand\tablename{表}
\fi
}
\@ifpackageloaded{float}{}{\usepackage{float}}
\floatstyle{ruled}
\@ifundefined{c@chapter}{\newfloat{codelisting}{h}{lop}}{\newfloat{codelisting}{h}{lop}[chapter]}
\floatname{codelisting}{コード}
\newcommand*\listoflistings{\listof{codelisting}{コード一覧}}
\makeatother
\makeatletter
\@ifpackageloaded{caption}{}{\usepackage{caption}}
\@ifpackageloaded{subcaption}{}{\usepackage{subcaption}}
\makeatother
\makeatletter
\@ifpackageloaded{tcolorbox}{}{\usepackage[skins,breakable]{tcolorbox}}
\makeatother
\makeatletter
\@ifundefined{shadecolor}{\definecolor{shadecolor}{rgb}{.97, .97, .97}}
\makeatother
\makeatletter
\makeatother
\makeatletter
\makeatother
\ifLuaTeX
\usepackage[bidi=basic]{babel}
\else
\usepackage[bidi=default]{babel}
\fi
\babelprovide[main,import]{japanese}
% get rid of language-specific shorthands (see #6817):
\let\LanguageShortHands\languageshorthands
\def\languageshorthands#1{}
\ifLuaTeX
  \usepackage{selnolig}  % disable illegal ligatures
\fi
\usepackage[]{natbib}
\bibliographystyle{jecon}
\IfFileExists{bookmark.sty}{\usepackage{bookmark}}{\usepackage{hyperref}}
\IfFileExists{xurl.sty}{\usepackage{xurl}}{} % add URL line breaks if available
\urlstyle{same} % disable monospaced font for URLs
\hypersetup{
  pdftitle={戦争の原因},
  pdfauthor={土井翔平},
  pdflang={ja},
  colorlinks=true,
  linkcolor={NavyBlue},
  filecolor={Maroon},
  citecolor={NavyBlue},
  urlcolor={NavyBlue},
  pdfcreator={LaTeX via pandoc}}

\title{戦争の原因}
\usepackage{etoolbox}
\makeatletter
\providecommand{\subtitle}[1]{% add subtitle to \maketitle
  \apptocmd{\@title}{\par {\large #1 \par}}{}{}
}
\makeatother
\subtitle{国際公共政策学}
\author{土井翔平}
\date{2023-04-21}

\begin{document}
\maketitle
\ifdefined\Shaded\renewenvironment{Shaded}{\begin{tcolorbox}[sharp corners, enhanced, interior hidden, borderline west={3pt}{0pt}{shadecolor}, frame hidden, breakable, boxrule=0pt]}{\end{tcolorbox}}\fi

\hypertarget{ux306fux3058ux3081ux306b}{%
\section*{はじめに}\label{ux306fux3058ux3081ux306b}}
\addcontentsline{toc}{section}{はじめに}

なぜ戦争は(社会的に望ましくないのに)起こるのか?

\begin{enumerate}
\def\labelenumi{\arabic{enumi}.}
\tightlist
\item
  \textbf{⼈間}は本能的に争うから。戦争を好きな⼈間が政治的指導者になるから。

  \begin{itemize}
  \tightlist
  \item
    確かに、(少なくとも一部の)人間は好戦的な性格を持ち、戦争を行うこと自体が目的である場合もある。
  \item
    戦争が起こった場合=当事者の政治家や⺠族が戦争好きだった(循環論法)
  \item
    安全保障政策\(\leadsto\)好戦的な人間を排除、性格を変える(生産的?)。
  \end{itemize}
\item
  \textbf{アナーキー}な国際社会では戦争をしても罰せられないから。戦争を防ぐアクターがいないから。

  \begin{itemize}
  \tightlist
  \item
    確かに、警察や司法のように違法行為を取り締まって、処罰する存在はいない。
  \item
    犯罪も全て抑止されていない。主権国家内でも内戦は起こる。
  \item
    安全保障政策\(\leadsto\)アナーキーな国際社会を変えて、世界規模の中央集権的政治体制を構築する(現実的?)
  \end{itemize}
\end{enumerate}

いずれにせよ、人間の本性や国際社会の構造だけでは、特定の時代や地域で戦争や暴力が多いことを説明できない。

\(\leadsto\)なぜ戦争が選択されるのか?

\hypertarget{ux6226ux4e89ux306eux69cbux9020}{%
\section{戦争の構造}\label{ux6226ux4e89ux306eux69cbux9020}}

\hypertarget{ux653fux6cbbux306eux624bux6bb5ux3068ux3057ux3066ux306eux6226ux4e89}{%
\subsection{政治の手段としての戦争}\label{ux653fux6cbbux306eux624bux6bb5ux3068ux3057ux3066ux306eux6226ux4e89}}

\hypertarget{ux6291ux6b62}{%
\subsection{抑止}\label{ux6291ux6b62}}

\hypertarget{ux4ea4ux6e09}{%
\subsection{交渉}\label{ux4ea4ux6e09}}

\hypertarget{ux6226ux4e89ux306eux539fux56e0}{%
\section{戦争の原因}\label{ux6226ux4e89ux306eux539fux56e0}}

\hypertarget{ux60c5ux5831ux306eux975eux5bfeux79f0ux6027}{%
\subsection{情報の非対称性}\label{ux60c5ux5831ux306eux975eux5bfeux79f0ux6027}}

\hypertarget{ux30b3ux30dfux30c3ux30c8ux30e1ux30f3ux30c8ux554fux984c}{%
\subsection{コミットメント問題}\label{ux30b3ux30dfux30c3ux30c8ux30e1ux30f3ux30c8ux554fux984c}}

\hypertarget{ux5206ux5272ux4e0dux53efux80fdux6027ux3068ux30eaux30b9ux30afux614bux5ea6}{%
\subsection{分割不可能性とリスク態度}\label{ux5206ux5272ux4e0dux53efux80fdux6027ux3068ux30eaux30b9ux30afux614bux5ea6}}

\hypertarget{ux6226ux4e89ux306eux7d42ux7d50}{%
\section{戦争の終結}\label{ux6226ux4e89ux306eux7d42ux7d50}}


\renewcommand\refname{補論:交渉可能領域の存在}
  \bibliography{references.bib}


\end{document}
